\title{CSCI 6150, Project Report}
\author{
        Caleb Adams \\
                Department of Computer Science\\
}
\date{\today}

% https://oeis.org/wiki/List_of_LaTeX_mathematical_symbols

\documentclass[12pt]{article}

\usepackage{graphicx}
\usepackage{subcaption}
\usepackage{tikz}
\usepackage[margin=1.5cm]{geometry}
\usepackage[most]{tcolorbox}
\usepackage{chessboard}
\usepackage{float}
\usepackage{hyperref}
%\usepackage{fontspec}
\usepackage{siunitx}
\usepackage{amsfonts}
\usepackage{amsmath}
\usepackage{amssymb}

\newcommand*\circled[1]{\tikz[baseline=(char.base)]{
            \node[shape=circle,draw,inner sep=2pt] (char) {#1};}}

\storechessboardstyle{4x4}{maxfield=d4}
\usetikzlibrary{arrows,automata}

\newtcblisting{commandshell}{colback=black,colupper=white,colframe=black!75!black,
listing only,listing options={language=sh},
every listing line={\textcolor{green}{\small\ttfamily\bfseries Caleb Adams \$> }}}

\begin{document}
\maketitle

\section{Introduction}

The goal of this project is to use material learned in the CSCI 6150 class to simulate
a Cart Pole system. The most heavily referenced materials for this project were the
Underactuated Robotics course materials from MIT \cite{underactuated} and the various
papers of Mattew Peter Kelly \cite{mattkelly}, though additional materials were used.

\begin{figure}[!ht]
 \centering
 \includegraphics[width=0.4\textwidth]{img/diagram1}
 \caption{A diagram of the Cart Pole system, which is a pendulum attached to a rolling cart.\cite{mattkelly}}
 \label{fig:oof}
\end{figure}

The Cart Pole system consists of a Cart, which is free to roll in the x directions, and a
pendulum, which is free to rotate around the center of the cart. The thing that makes the
Cart pole tricky is the interplay between these two phenomenon.

\newpage

\section{Dynamics of the Cart Pole}

A Balance of forces can be see on the Cart Pole in Figure 2. This balance of forces
is what is used to generate the equations which govern the Cart Pole.

\begin{figure}[!ht]
    \centering
    \begin{subfigure}[b]{0.3\textwidth}
        \includegraphics[width=\textwidth]{img/diagram2}
        \caption{The freebody diagrams describing the Cart Pole.\cite{mattkelly}}
        \label{fig:asd123}
    \end{subfigure}
    ~ %add desired spacing between images, e. g. ~, \quad, \qquad, \hfill etc.
      %(or a blank line to force the subfigure onto a new line)
    \begin{subfigure}[b]{0.3\textwidth}
        \includegraphics[width=\textwidth]{img/diagram3}
        \caption{Diagram of the Symbols used for the Cart Pole system.}
        \label{fig:qwerty}
    \end{subfigure}
    \caption{Diagrams of the Cart Pole}\label{fig:asdfasdf}
\end{figure}

From these free body diagrams, we can generate 3 equations (the force balance on the cart,
the balance on the pole, and the balance around the pivot):

\begin{equation} \label{eq:1}
(F - T) \hat{i} + (N - W_1 - T_y) \hat{j} = m_1 \ddot{p_1}
\end{equation}

\begin{equation} \label{eq:2}
(T_x) \hat{i} (T_y - W_2 ) \hat{j} = m_2  \ddot{p_2}
\end{equation}

\begin{equation} \label{eq:3}
(T_x) \hat{i} (T_y - W_2 ) \hat{j} = m_2  \ddot{p_2}
\end{equation}

In the end, these equations give us the following second order linear system to
discribe the cart pole \cite{mattkelly} \cite{underactuated}:

\begin{equation} \label{eq:4}
\begin{bmatrix}
\cos \theta & l \\
m_1 + m_2 & m_2 l \cos \theta
\end{bmatrix}
\begin{bmatrix}
\ddot{x} \\
\ddot{\theta}
\end{bmatrix}
=
\begin{bmatrix}
- g \sin \theta \\
F + m_2 l \dot{\theta}^2 \sin \theta
\end{bmatrix}
\end{equation}

For the purpose of this project I transform this system back into two fuctions.
This system can be viewed as $A x = b$, so simply $x = A^{-1} b$ will give us
a solution. Then, we can solve for the second derivatives. Doing this yeilds the following
two equations:

\begin{equation} \label{eq:5}
\ddot{x} = \frac{-g \sin \theta m_2 l \cos \theta + (-l)(F + m_2 l \dot{\theta}^2 \sin \theta) }{ (\cos \theta)^2 m_2 l - l(m_1 + m_2) }
\end{equation}

\begin{equation} \label{eq:6}
\ddot{\theta} = \frac{-(m_1 + m_2)(-g \sin \theta) + \cos \theta (F + m_2 l \dot{\theta}^2 \sin \theta) }{ (\cos \theta)^2 m_2 l - l(m_1 + m_2) }
\end{equation}


\section{Methods Used \& Implementation}

In general, this was formulated as an Initial Value Problem (IVP), solved accordingly,
and then simulated using OpenGL.

\subsection{Formulation as a System of ODEs}

This problem was solved using techniques found in chapter 11 of Numerical mathematics
and Computing. First, The variables are substituted and make into a system of first
order equations \cite{classbook}.

\[
X =
\begin{bmatrix}
  x_1 = x \\
  x_2 = \dot{x} \\
  x_3 = \theta \\
  x_4 = \dot{\theta}
\end{bmatrix}
,
\hspace{2mm}
X(0) =
\begin{bmatrix}
  x_1 = 0 \\
  x_2 = 0 \\
  x_3 = \pi / 4 \\
  x_4 = 0
\end{bmatrix}
\]

Then, we note that we can discribe the state vector $X$'s derivative vector $\dot{X}$,
which is what we can use to solve an IVP with the system. Note that this is where we want
to jump back to the separated equations 5 and 6 in part 2. These equations will not take the forms:

\begin{equation} \label{eq:7}
F(x_3, x_4) = \frac{-g \sin x_3 m_2 l \cos x_3 + (-l)(F + m_2 l x_4^2 \sin x_3) }{ (\cos x_3)^2 m_2 l - l(m_1 + m_2) }
\end{equation}

and ...

\begin{equation} \label{eq:8}
G(x_3, x_4) = \frac{-(m_1 + m_2)(-g \sin x_3) + \cos x_3 (F + m_2 l \dot{x_4}^2 \sin x_3) }{ (\cos x_3)^2 m_2 l - l(m_1 + m_2) }
\end{equation}

Then we form the derivative vector $\dot{X}$:

\[
\dot{X} =
\begin{bmatrix}
\dot{x_1} \\
\dot{x_2} \\
\dot{x_3} \\
\dot{x_4}
\end{bmatrix}
=
\begin{bmatrix}
x_2 \\
F(x_3, x_4) \\
x_4 \\
G(x_3, x_4)
\end{bmatrix}
\]

This then allows us to form the simplest IVP, using Euler's method with a stepsize $h$:

\begin{equation} \label{eq:9}
X(t+1) = X(t) + h \cdot \dot{X}(X(t),t)
\end{equation}

This also allows for a Modified Euler's method with stepsize $h$:

\[
K_1 = \dot{X}(X(t),t)
\]
\[
K_2 = \dot{X}(X(t+1) + h \cdot K_1,t+1)
\]
\begin{equation} \label{eq:10}
X(t+1) = X(t) + h \frac{K_1 + K_2}{2}
\end{equation}

And finally, this allows for an RK4 method to be used, here shown with stepsize $h$:

\[
K_1 = \dot{X}(X(t),t)
\]
\[
K_2 = \dot{X}(X(t)+ \frac{h}{2} K_1,t + \frac{h}{2})
\]
\[
K_3 = \dot{X}(X(t)+ \frac{h}{2} K_2,t + \frac{h}{2})
\]
\[
K_4 = \dot{X}(X(t)+ h K_3,t + h)
\]

\begin{equation} \label{eq:10}
X(t+1) = X(t) + h \frac{K_1 + 2K_2 + 2K_3 + K_4}{6}
\end{equation}

These three methods, Euler, Modified Euler, and RK4 are Implemented in this project. These are
methods in the Cart Pole Object within the \texttt{sim\_cartpole.cpp} file. Additionally,
the Euler method is implemented for some simple pendulums within the \texttt{sim\_pendulum.cpp}
file.

\subsection{OpenGL and Rendering}

\begin{figure}[!ht]
    \centering
    \begin{subfigure}[b]{0.3\textwidth}
        \includegraphics[width=\textwidth]{img/carts1}
        \caption{Example Cart Poles.}
        \label{fig:gull}
    \end{subfigure}
    ~ %add desired spacing between images, e. g. ~, \quad, \qquad, \hfill etc.
      %(or a blank line to force the subfigure onto a new line)
    \begin{subfigure}[b]{0.3\textwidth}
        \includegraphics[width=\textwidth]{img/carts2}
        \caption{Example Cart Poles.}
        \label{fig:tiger}
    \end{subfigure}
    ~ %add desired spacing between images, e. g. ~, \quad, \qquad, \hfill etc.
    %(or a blank line to force the subfigure onto a new line)
    \begin{subfigure}[b]{0.3\textwidth}
        \includegraphics[width=\textwidth]{img/pend}
        \caption{The test pendulums.}
        \label{fig:mouse}
    \end{subfigure}
    \caption{Screenshots of simulated systems}\label{fig:animals}
\end{figure}


\section{Results \& Future Work}

\subsection{Simulation Results}

Overall the results are quite nice. The standard Euler method works well and transfers
energy and momentum throughout the system as I would expect. However, it was difficult to
judge if results were correct, as the cart pole is a chaotic system. The changes in types
of numerical methods would make the chatoic nature of the system apearent after only a
few iterations, and that makes it hard to directly compare methods. Additionally,
the lack of a clear closed loop solution to the cart pole makes it difficult to compare
each individual result to a ground truth.

\begin{figure}[!ht]
 \centering
 \includegraphics[width=0.7\textwidth]{../data/Figure_1}
 \caption{Shows the displacement in the $x$ direction (y axis) of the cart
 over time $t$ (x axis). Green is Euler's method, Red is the Modified Euler's method, and Blue is RK4.}
 \label{fig:yeet}
\end{figure}

\begin{figure}[!ht]
 \centering
 \includegraphics[width=0.7\textwidth]{../data/Figure_2}
 \caption{Shows the angle of the pole of the cart in radians (y axis) over
 time $t$ (x axis). Green is Euler's method, Red is the Modified Euler's method, and Blue is RK4.}
 \label{fig:yub}
\end{figure}

Many simulations, or models used for Trajectory Optimization, do not account for a dampening
factor or friction on the cart system. I attempted to model these, which is clear in Figures
4 and 5 as the amplitude clearly decreases over time. I do not think I adiquately modeled
these dampening and friction forces, but as I mentioned above it is hard to tell due to the
chaotic nature of the Cart Pole. I would at least expect the rate of dampening to be similar
in each method, which is not something I see and leads me to believe there is some error associated
to that.

\subsection{Future Work}

\bibliographystyle{IEEEtran}
\bibliography{sources}

\end{document}



























% yee
